
\documentclass[a4paper]{TUBAFprotokoll}	
\usepackage[ngerman]{babel}
\usepackage[utf8]{inputenc}
\usepackage[T1]{fontenc}
\usepackage{chemmacros}	
\usepackage{graphicx}
\usepackage{setspace}
\usepackage{geometry}
\usepackage{multicol}
\usepackage{amsmath}
\usepackage{upgreek}
\usepackage{chemfig}
\usepackage{ghsystem}
\usepackage{float}
\usepackage{longtable}
\usepackage{booktabs}

\usepackage{hyperref}
\chemsetup{modules={all}}


\hbadness=10000
\vbadness=10000
\sloppy

\TUBAFFakultaet{Fakultät für Werkstoffwissenschaften}
\TUBAFInstitut{Institut für Elektronik- und Sensormaterialien}
\TUBAFPraktikum[HNSt]{Herstellung von Nanostrukturen}
\TUBAFVersuch[1]{Ferrofluid}
\TUBAFPraktikant{Gruppe C: Meike Bösel, Niclas Krause}
\TUBAFStudiengang[NT]{Diplom Nanotechnologie}
\TUBAFSemester{3.}
\TUBAFModul[HNST]{Herstellung von Nanostrukturen
	\TUBAFPraktikumstag{12.12.2021}
	\TUBAFBetreuer{Reichel, Oestreich}
	
	\begin{document}
		\maketitle
		\tableofcontents
		\newpage
		\section{Aufgabenstellung}
		Im Praktikum sind superparamagnetische Magnetit-Nanopartikel aus Eisenchlorid und wässriger Ammoniaklösung herzustellen. Die Nanopartikel haben einen Durchmesser von ca. 10 Nanometern und bilden eine stabile kolloidale Suspension in der Trägerflüssigkeit, ein sogenanntes Ferrofluid. Um die Agglomeration der Partikel zu verhindern, werden sie oberflächenfunktionalisiert. Der Nachweis der Nanoskaligkeit der Partikel erfolgt durch den Rosensweig-Effekt: Beim Anlegen eines äußeren Magnetfeldes richten sich die Nanopartikel entlang der Feldlinien des Magnetfeldes aus, sie lassen sich aber nicht dauerhaft magnetisieren.
		
		\section{Grundlagen}
		\subsection{Ferrofluide}
		Ein Ferrofluid ist eine stabile kolloidale Suspension bestehend aus magnetischen Nanopartikeln mit einem Durchmesser von 10 Nanometern in einer wässrigen oder öligen Phase. Die Nanopartikel, die Ferrofluide beinhalten müssen, sind Ferro-, Ferri- oder Antiferromagnetika auf Basis von Ferriten oder Granaten. Diese Nanopartikel können dann durch ein externes Magnetfeld magnetisiert und ausgerichtet werden. Bei Überschreitung der kritischen Feldstärke entstehen charakteristische Igelstrukturen, die als Rosensweig-Effekt bezeichnet werden. Dieser Effekt tritt auf als Resultat des Gleichgewichts aus Oberflächenspannung, Gravitation und der magnetischen Kraft. Dabei versucht das Ferrofluid die Energie zu minimieren, was durch Ausrichtung der Dipole im Material erreicht wird. Dieser Effekt ist nur bei Ferrofluiden zu sehen, da bei magnetheologischen Lösungen bei hinzuführen eines externen Magnetfeldes eine Verfestigung entsteht. 
		
		\subsection{Eigenschaften der Ferrofluide}
		Ferrofluide zeigen aufgrund des Superparamagnetismus weder Hysterese noch Remanenz auf, demzufolge weisen sie auch unter der Curie-Temperatur keine bleibende Magnetisierung auf. Die Ursache für dieses Phänomen ist die Neel-Relaxation und die Brown-Relaxation. Dabei orientieren sich nach thermischer Anregung die magnetischen Teilchen ständig ohne äußere Einflüsse. Die Zeit bis sich ein Teilchen komplett gedreht hat bezeichnet man dabei als Neel-Reaktionszeit. Dabei verhält sich eine Ansammlung dieser Teilchen makroskopisch wie ein Paramagnet, allerdings besitzt es trotzdem eine hohe magnetische Sättigung. Daher liegt die Magnetisierbarkeit superparamagnetischer Materialien zwischen der von Para- und der von Ferromagneten.\\
		Zudem sollten Ferrofluide eine hinreichende Stabilität gegen Van-der-Waals-Kräfte aufweisen um eine Agglomeration zu verhinden. Dabei solte bei kleinen Abstände, da dort die Kräfte sehr groß werden, eine Stabilisierung der Partikeloberfläche erfolgen. Dabei kann diese Stabilisierung sterisch oder auch elektrostatisch funktionieren. Dabei wird in diesem Versuch als Dispergirhilfsmittel TMAH-Lösung verwendet. Ferrofluide selbst weisen ebenfalls Stabilität gegen magnetische Agglomeration, dabei verhindert die thermische Bewegung Kettenstrukturen und die Segregation durch Feldgradienten. \\
		Eine weitere Eigenschaft der Ferrofluide ist die veränderbare Viskosität. Dabei können die Teilchen durch Scherstömungen in Rotation versetzt werden, dabei wird das magnetische Moment aus der Magnetfeld-Richtung gedreht und führt so zu einer höheren Viskosität durch orientierte Nanopartikel-Kettenstrukturen.
		
		\subsection{Bottom-Up Synthese}
		In diesem Versuch wird die Bottom-Up Synthese angewendet, um die Nanopartikel herzustellen. Diese Synthese funktioniert wie der Name bereits sagt durch den Aufbau von den Nanoteilchen von noch kleineren Teilchen. In diesem Versuch wird dafür die chemische Reaktion von Eisen(II)- und Eisen(III)chlorid und Natronlauge zu Magnetit und Ammoniumchlorid verwendet. \\
		2FeCl_{3} + FeCl_{2} + 8 NH_{3} + 4 H_{2}O $\rightleftharpoons$ Fe_{3}O_{4} + 8 NH_{4}Cl
		Dabei entsteht der superparamagnetische Magnetit, der für das Ferrofluid benötigt wird und Ammoniumchlorid, welches mit seinen basischen und polaren Eigenschaften für die Oberflächenstabilisierung verwendet wird.\\ Dabei müssen die Teilchen die entstehen immer nur einen Weißschen Bezirk beinhalten, da so die Magnetische Ausrichtung besonders stark ist.
		
		\subsection{Verwendung von Ferrofluiden}
		Ferrofluide finden Anwendung in verschiedensten Bereichen. Beispielsweise werden sie als Dichtungsmaterialien an rotierenden Wellen eingesetzt, indem sie von Permanentmagneten an der abzudichtenden Stelle gehalten werden und so große Drücke ertragen. Außerdem werden sie aufgrund der veränderbaren Viskosität als Dämpfungmittel oder Kühlflüssigkeit in Lautsprechern aber auch als Druck-, Neigungs- und Beschleunigungssensoren engesetzt. Auch in der Medizin finden sich Anwendungen für Ferrofluide. Beispielsweise in der Krebstherapie, wo die Ferrofluide in das Tumorgewebe eingebracht wird und durch ein magnetisches Wechselfeld Erwärmung erzielt wird und den Tumor durch Überhitzung zerstört. Dieser Prozess wird auch als Hyperthermie bezeichnet. Auch werden sie durch das Festhalten von angekoppelten Medikamenten in bestimmten Körperregionen für Drug Targeting verwendet.\\
		\\
		\section{Auswertung}
		Der Rosensweig-Effekt konnte nicht direkt nach Abschluss des Versuchs beobachtet werden.So reagierte das Fluide zwar auf das Magnetfeld und wurde von diesem angezogen doch es kamm zu keiner Ausbildung der typischen Igelform. So sammelte sich das Fluide nur in einer Halbkugelform an der Stelle wo der Magnet die Petrischale berührte. Weiterhin ließ sich keine Veränderungen der Viskosität feststellen als das Fluide mit dem Magnetfeld in Kontakt kam. Genau so gab es keine Auflösung der Suspension. Somit ist ausgeschlossen das durch dass Experiment eine magnethorheologische Flüssigkeit entstanden ist.\\
		So ließ sich aber festellen das dass erstellte Fluide deutlich flüssiger war als andere Ferrofluide. Das lässt vermuten das im erstellten Fluide noch etwas Wasser von den Waschungen zurückgeblieben ist. Im zuge dieser Vermutung ließen wir das Fluide für ein paar Tage offen stehen sodass das Wasser verdunsten konnte. So wurde das Fluide über ein paar Tage dickflüssiger und war dann auch in der Lage den Rosensweig-Effekt auszubilden. Hier konnte nun die typische Igelform beobachtet werden. Dies war aber im Vergleich zu einer ältern Probe relativ schwach ausgeprägt. Das liegt vermutlich daran das immernoch etwas Wasser in unserem Ferrofluide verhanden war. So sollte sich der Rosensweig-Effekt mit noch mehr Zeit zum verdunsten des Wassers noch stärker ausprägen.\\
		Bei deutlich älteren Proben ist die gesamte Flüssigkeit verdunstet, wodurch keine Suspension mehr vorliegt und nur noch die Nanopartikel zurückblieben.\\
		Weiterhin war keine dauerhafte Magnetisierung zu beobachten. Nach dem Entfernen des Magneten verhielt sich das Ferrofluid wieder wie eine normale Flüssigkeit. Dies alles ist für Ferrofluide typisch und liegt daran, dass sich die superparamagnetische Parrtikel nicht dauerhaft magnetisieren lassen. Somit ist auch bewiesen das in diesem Versuch ein Ferrofluid erzeugt wurde.\\
		\\
		\subsection{Fehlerbetrachtung}
		Wie bereits beschrieben war das Fluid am Anfang nicht in der Lage den Rosensweig-Effekt auszubilden. Dies lag vermutlich daran das Wasser im Fluid zurückgeblieben ist was zugleich die flüssigere Konsistenz im Vergleich mit anderen Ferrofluiden erklären würde. Dieses Wasser ist höchstwahrscheinlich beim waschen der Nanopartikel in der Petrischale zurückgeblieben. So sollte in zukünftigen Versuchen darauf geachtete werden so viel Wasser wie möglich zu entfernen oder wenn nötig dieses am ende verdunsten zu lassen.\\
		So zeigt der Umstand das älter Proben einen stärkeren Rosenweig-Effekt aufweisen das es eine idealle Konsistenz für Ferrofluide gibt. Weiterhin scheint der Spielraum relativ klein zu sein sodas nur geringe Wassermengen zu viel oder zu wenig das Ergebnis stark beeinflussen.   
	\end{document}