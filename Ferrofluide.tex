\documentclass[a4paper]{TUBAFprotokoll}	
\usepackage[ngerman]{babel}
\usepackage[utf8]{inputenc}
\usepackage[T1]{fontenc}
\usepackage{chemmacros}	
\usepackage{graphicx}
\usepackage{setspace}
\usepackage{geometry}
\usepackage{multicol}
\usepackage{amsmath}
\usepackage{upgreek}
\usepackage{chemfig}
\usepackage{ghsystem}
\usepackage{float}
\usepackage{longtable}
\usepackage{booktabs}

\usepackage{hyperref}
\chemsetup{modules={all}}


\hbadness=10000
\vbadness=10000
\sloppy

\TUBAFFakultaet{Fakultät für Werkstoffwissenschaften}
\TUBAFInstitut{Institut für Elektronik- und Sensormaterialien}
\TUBAFPraktikum[HNSt]{Herstellung von Nanostrukturen}
\TUBAFVersuch[1]{Ferrofluid}
\TUBAFPraktikant{Gruppe ?: Meike Bösel, Niclas Krause}
\TUBAFStudiengang[NT]{Diplom Nanotechnologie}
\TUBAFSemester{3.}
\TUBAFModul[HNSt]{Herstellung von Nanostrukturen
\TUBAFPraktikumstag{12.12.2021}
\TUBAFBetreuer{}